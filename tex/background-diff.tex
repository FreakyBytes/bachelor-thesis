% !TeX spellcheck = en_US
\section{Data formats for System Biology models}
	\begin{itemize}
		\item formats
		\subitem SBML
		\subitem CellML
		\subitem \sedml
		\item everything in XML
		\item semantic annotations
	\end{itemize}
	\todo{look at STATS paper/MOST for version statistics}

\section{Detecting differences in Version Control Systems}
	\subsection{Unix Diff}
	\label{sec:background:diff:unix-diff}
	The essential building block for each Version Control System (VCS) is an algorithm or tool to detect differences between two or more version, because just storing each version complete is a waste of storage and bandwidth. Further processing change sets enables the system to automatically merge different development branches and therefore making collaboration more efficient.
	
	Nowadays all general-purpose VCS use the \texttt{diff} utility, developed as part of UNIX. It is based on the Hunt-McIlroy algorithm, which tries to solve the longest common subsequence problem for two files. \cite{Hunt1976}
	The result is a report of all differences between two files, "expressed as a minimal list of line changes to bring either file into agreement with the other" \cite{Hunt1976}
	
	\begin{comment}
	\begin{itemize}
		\item Based on solving the longest common subsequence problem
		\item "The program diff reports differences between two files, expressed as a minimal list of line changes to bring either file into agreement with the other" \cite{Hunt1976}
		\item "The central algorithm of diff solves the ‘longest common subsequence problem’ to find the lines that do not change between files" \cite{Hunt1976}
	\end{itemize}
	\end{comment}
	
	\subsection{XML Diff}
	\label{sec:background:diff:xml-diff}
	The problem with general-purpose VCS or more specific the difference algorithm used by them (cf. section \ref{sec:background:diff:unix-diff}) is however, that they do not make any assumptions about the underlying document and therefore fail to recognize feature specific to the format. These line-base algorithms proved to be problematic especially for \xml, since line breaks can be neglected, without changing the encoded information. \cite{Ronnau2005}
	It can be concluded, that the usual diff algorithm is too fine granular, since it also reacts to changes in the encoding. Generally \xml can be represented as hierarchical tree structure and thus it is possible to detect changes based on their position in the tree, instead by a line-number. \cite{Wang2003,Chawathe1996,Cobena2002}
	This can be archived by identifying unchanged subtrees and map them between source and destination version. Going out from those mapped tree, more and more nodes can be matched, under consideration of ancestors, descendants, and labels. Also changed attributes or text-nodes need to be treated differently. \cite{Cobena2002}
	
	\todo{add ref to both examples XyDiff + the other one?}
	
	
	\begin{comment}
	
	cf. Cobena2002 \cite{Cobena2002} and \cite{Waltemath2013} in section 2.2.1
	Chawathe et al., 1996 \cite{Chawathe1996}
	\begin{itemize}
		\item "With flat information deltas may be represented simply as sets of tuples or records inserted into, deleted from, and updated in relations. In hierarchical information, we want to identify changes not just to the 'nodes' in the data, but also to their relationships." \cite{Chawathe1996}
		\item "General-purpose version control systems attempt to handle any kind of document and thus make no assumptions about the underlying document format. Usually, those systems distinguish between binary documents and text documents. Appropriate diff algorithms are employed to detect changes between two versions of a document. Those diff algorithms are either line based (for text documents) or byte based (for binary documents). General-purpose version control systems provide good control for line-organized text documents such as source code or latex documents. For XML documents, where the organization into lines can be neglected, 2 line-based version control is inappropriate." \cite{Ronnau2005}
		\item requirements
			\subitem "Within a delta, the exact location of the change must be identified." \cite{Ronnau2005}
			\subitem "XML Structure Information. An XML document is generally a hierarchically structured document, and can be represented in a tree structure. However, an XML document has other features that distinguish it from a general labeled tree. X-Diff introduces the	notion of node signature and a new matching between the trees corresponding to the two versions of a 	document. Together, these two features are used to find the minimum-cost matching and generate a minimum-cost edit script that is capable of transforming the original version of the document to the new version." \cite{Wang2003}
			\subitem "Unordered Trees. Since XML documents can be represented as trees, the change detection problem is related to the problem of change detection on trees." \cite{Wang2003}
		\item "It [the algorithm] tries to detect (large) subtrees that were left unchanged between the old and new versions. These are matched. Starting from there, the algorithm tries to match more nodes by considering ancestors and descendants of matched nodes and taking labels into consideration. Our algorithm also takes advantage of the specificities of XML data. For instance, it knows of attributes and attribute updates and treat them differently from element or text nodes. It also takes into account ID attributes to match elements." \cite{Cobena2002}
	\end{itemize}
	\end{comment}
	
	\subsection{BiVeS}
	Continuing on the problem described, in section \ref{sec:background:diff:xml-diff}, the same issue with \xml documents and normal delta algorithms scales to more specific documents and \xml delta algorithms. In this case we look at \xml encoded biological models, either using \cellml or \sbml. Even though algorithms described in the prior section are treating \xml documents as trees, they still lack deeper understanding of a biological model for instance. \bives is addressing this issue by building upon XyDiff (cf. section \ref{sec:background:diff:xml-diff}), but also incorporating additional features specific to system biological models, like ontology links, attribute references, and special parts of the model, which are treated atomically. \cite{Scharm2015}
	
	
	\begin{itemize}
		\item benefits of XyDiff compared to unix-diff
		\subitem problems with XML
		\subitem no deeper "understanding"
		\item cf. \cite{Waltemath2013} (Oxford 2012), \cite{Scharm2015}
		\item algorithm (list of quotes from \cite{Scharm2015})
		\subitem two versions of an XML-encoded model are translated into an internal tree structure. For every node n in the tree, a hash sum n r and a weight n x are calculated 
		\subitem The weight of a node is thus always greater than the weight of its children. As such, the weight represents the size of the corresponding subtree 
		\subitem The hash sum of a node n represents the signature of the subtree rooted at n 
		\subitem While n r unambiguously defines the subtree rooted in n, n r does not need to be unique among all nodes in the tree. Thus, if n r ¼ m r then the subtrees in n and m are identically equal 
		\subitem First, nodes are being mapped with respect to their identifiers 
		\subitem id attributes in the XML documents serve as identifiers. In addition, we also evaluate biological identifiers, specifically links into bio-ontologies 
		\subitem Second, the initial mapping is propagated upwards into the trees. 
		\subitem The connections of a node’s children are evaluated in a depth-first traversal of T 2 . If a node n in T 2 is connected to a node m in T 1 then a mapping of parent ðnÞ to parent ðmÞ is suggested 
		\subitem If, in contrast, n is not connected, we examine the candidates that were previously suggested by the connections of n’s children. 
		\subitem Candidates which have a different tag name than n and candidates which al- ready have a connection are neglected. 
		\subitem Among the remaining candidates, the algorithm chooses the one that received the best suggestions and connects it to n 
		\subitem Third, the algorithm makes use of the initially computed signatures and maps nodes of T 2 on nodes of T 1 
		\subitem A priority queue U is maintained to sort the nodes of T 2 based on their weights. Initially, U only consists of the root node of T 2 . 
		\subitem Unless U is empty, the algorithm repeatedly removes node n 2 U  T 2 with the biggest weight, which represents the biggest subtree in the queue 
		\subitem Fourth, the algorithm improves the quality of the mapping by examining the network structure of T 1 and T 2 in a top-down approach. For every mapping n 2 T 2 on m 2 T 1 , it compares unmatched children of n and m to find missed mappings 
		\subitem The algorithm evaluates the matrix greedily and adds new mappings up to a maximum distance of 0.9. Thus, nodes which have nothing in common will not be connected 
		\subitem Additional mapping rules capture the domain characteristics of the processed data. Following the current specifications for SBML and CellML, we prohibit certain changes in the hierarchical tree of document nodes. Specifically, we treat parts of the model as atomic con- structs for which we define restrictions on possible network operations 
		\subitem This step is a major reason why our algorithm outperforms standard XML diff algorithms. 
		\subitem insert if an entity is present in T 2 but absent in T 
		\subitem delete if an entity is present in T 1 but absent in T 
		\subitem move if a node is present in both documents, but either (i) the parents in the corresponding trees are not connected or (ii) the parents are connected, but the sequence of their siblings has changed 
		\subitem update if the value of an attribute, a text node's content or the tag name of a node was modified
		\subitem After the mapping, we distinguish two types of nodes: mapped nodes and unmapped nodes. Unmapped nodes n 2 T 1 [ T 2 are nodes for which the algorithm could not find a matching node in the opposite tree. These nodes and their attributes correspond to either inserts or deletes, depending on their origin 
		\subitem In contrast, mapped nodes are nodes for which the algorithm did find a matching node in the opposite tree. If the parents of such a mapping of n 2 T 2 onto m 2 T 1 are not connected, or if the se- quence among their siblings has changed, then these nodes are included in the set of moves 
	\end{itemize}
	
\section{Managing Versions}
	\subsection{Traditional Version Control Systems}
	\begin{itemize}
		\item benefits of version control systems
			\subitem \todo{elaborate}
			\subitem "The need for model version control has been previously discussed in research groups facing model evolution in computational biology (Beard et al., 2009; Cuellar et al., 2006; Hucka et al., 2010; Li et al., 2010; Miller et al., 2011). In general, VCSs such as Subversion (http://subversion.apache.org/) (SVN)" \cite{Waltemath2013}
		\item simple file storage
			\subitem storing files next to each other in the file system
			\subitem "-version1", "-version2", "-final"
			\subitem no meta information stored along with versions (author, time stamp)
			\subitem collaboration causes problems
			\subitem but: simple and quick
		\item SVN
			\subitem client/server architecture
			\subitem collaboration possible, through branching and merging capability
			\subitem reverse-delta storage
			\subitem no offline work possible
		\item GIT
			\subitem distributed VCS
			\subitem reverse-delta storage with version graph
			\subitem build for heavy collaboration by using information from the version graph for merging
			\subitem \todo{refer to version-graph later, when explaining db concept}
			\subitem \url{https://git-scm.com/book/en/v2/Git-Internals-Plumbing-and-Porcelain}
	\end{itemize}
		
	\subsection{Challenges traditional VCS face with Systems Biology models}
	\begin{itemize}
		\item SVN and GIT are using unix-diff -> not suitable for XML files (cf. XML diff section)
		\item no semantic information about the changes (cf. comodi)
		\item no domain specific knowledge to improve mapping of different XML-tree branches
			\subitem important for good detection of tree branch movement etc. 
			\subitem "A model VCS should be tailored to existing model representation formats, which are typically XML and RDF based. It should furthermore reflect the temporal evolution of a model and present model changes to the users." \cite{Waltemath2013}
			\subitem "These
			common changes are detected by the LCS algorithm, but they are in fact irrelevant for the model’s history and would be neglected by entity-based algorithms. In other words, although being successfully used for source code version control, LCS is not suitable for XML version control (Chawathe et al., 1996)." \cite{Waltemath2013}
		\item As proposed in \cite{Waltemath2013} there are 2 major approaches to keep track of changes:
			\subitem keeping track of (minor) changes in the model itself
			\subitem tracking versions in the repository
	\end{itemize}


\section{Ontologies in Computer Science}

	\begin{itemize}
	\item definition
		\subitem formal definition, properties and relation of entities
	\item use of BioOntologies cf. Courtot
	\end{itemize}

	\todo{cite owl standard, when explaining comodi import}
	\todo{http://msb.embopress.org/content/7/1/543.short for ontology overview}
	
	\subsection{\comodi}
	\label{sec:background:comodi}
	\begin{itemize}
		\item cf. \cite{Scharm2016}
		\item  They clarify the intended semantics of the data, which makes the data more accessible, sharable, and interoperable 
		\item An ontology is a tool to provide meaning to data, the information of which can then be subjected to algorithmic processing
		\item We believe that a similar approach should be taken for the semantic description of differences between versions of a model. Using the semantic layer to describe changes in a model allows for storing meaning together with possible implications of these changes.
		\item It is important to track these changes for a number of reasons. Changes in parametrisations or on the underlying network may lead to a situation where the original results are not reproducible anymore.
		\item With respect to simulation results, change records can help to predict modifications in the simulation outcome. Finally, a good communication of model changes increases the trust of scientists wishing to reuse a model for their own purposes.
		\item  as the list of changes increases it becomes harder to grasp their relevance. To address this problem, we present an ontology to annotate the changes identified with the BiVeS algorithm.
		\item  it is always possible to specify the XML entity that is subject to a change. 
		\item COMODI is organised into four branches around the central concept Change: XmlEntity, Intention, Reason, Target
		\item Intention and Reason both indicate the purpose of a change. On the one hand, the Intention specifies the aim of a change, particularly with respect to consequences in the future. In our example, the intention of modifying the parameter value is a Correction. On the other hand, a Reason specifically focuses on the cause of a change. In our example, a MismatchWithPublication caused an update of the parameter value.
		\item COMODI basically distinguishes between five layers in a model document, that can be subject to a change:
		\item Finally, different changes might be linked to each other if they have mutual dependencies.
		\item The COMODI ontology is specifically designed for the annotation of differences between versions of a computational model in the life sciences.
		\item Some information can directly be inferred and thus be annotated automatically with BiVeS
		\item  The ontology terms specify the type of change for each detected difference. Usually, a combination of COMODI terms from different branches is necessary to characterise a change sufficiently.
		\item COMODI terms can also be applied to differences detected between models in any other encoding format, including even code from proprietary languages such as MATLAB 
		\item COMODI cannot, however, be used to encode provenance, such as information about the user who changed the model or information about the tool used to update the file. It can, however, easily be coupled with ontologies for provenance.
	\end{itemize}