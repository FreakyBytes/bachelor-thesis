% !TeX spellcheck = en_US

Opposed to traditional VCSs (cf. Section \ref{sec:background:manage-versions:traditional-vcs}) the system presented in this thesis is not meant to support developers during the development process, but rather to assist other developers, who want to build on existing models.
This shift in focus, allows to analyze and organize already published models, as well as it does not need any interaction from the modelers. Second point might help to building up the data stock quicker, but also limits the amount of meta information.
By not relying on anybody specifically using this system, I was able to quickly accumulate a large test set of $14503$ model versions from $3367$ distinct models, filling the database with $4.307$ versions per model on average and consuming $8.5$ GBytes of disc storage.
This test set was generated on 2016-10-14 from all publicly available workspaces in \emph{PMR2} and all releases from \emph{BioModels Database} (cf. Section \ref{sec:background:modelrepo}) using the \modelcrawler (cf. Section \ref{sec:impl:modelcrawler}).

Being an analyzes platform and not a day-to-day tool sets different constraints. For instance access time becomes more crucial, whereby consumed drive space is less a concern.
My implementation takes full advantage of this loosened space restriction.
The storage structure, I decided on, does not make use of an efficient reverse-delta storage, as traditional VCS (cf. Section \ref{sec:background:manage-versions:traditional-vcs}). Instead each model is stored completely for multiple reasons: it allows for a very lose coupling to the base \masymos implementation, meaning the diff-extension is additive to the features of \masymos and does not require for intrusive changes of the code base. Further the core functionality of \masymos, providing a full text and structural search index, is not interrupted, so if required each version of a model can be treated as single instance. Resulting in easier structural analyses per model version and less expensive operations on the indexes, when inserting a new model version.

On the other hand the complete storage of each model increases the database size significantly. But to keep the scope of this work concise and the implementation reasonable, I focused on a good data interlinkage and less on a storage efficient data model.

This decision might show some drawbacks with growing amount of models and versions.
First issue to consider would be higher storage consumption. But also slower query execution should be considered.
Secondly \bives creates multiple changes, which have a single common source. For instance adding a \texttt{Species} will result in at least five changes. These include the insertion of the element itself and the insertion of the attributes \texttt{id}, \texttt{name}, \texttt{initialConcentration}, and \texttt{compartment} reference.
Storing this many borderline redundant changes is increasing the overall size dramatically.

A way to reduce the overhead of storing deltas, could be to calculate these deltas only on demand. This would, however, reduce the database to a cache for \bives and distinguish all possibilities to analyze changes in large scale.

Both issues could also be improved or even solved by using a database design with less redundancies. This means in relation to the first point, to implement the database as reverse-delta storage, accepting all the mentioned drawbacks.
In the second issue less redundancies could be archived by grouping changes, which clearly share the same origin.

Grouping in changes will inevitably lead to less accurate annotation with the \comodi ontology, which on the other hand will limit the use in analysis and statistics of changes. 
Then again, currently the \comodi ontology is only partly used in this project. The \texttt{Reason} and \texttt{Intention} branches (cf. Section \ref{sec:background:onto:comodi}) are not used, because they aim to explain the cause and purpose of a change. These aspects are hard to determine automatically just with a delta between two model versions, therefore is \bives not capable of using these annotations automatically.
To integrate also \texttt{Reason} and \texttt{Intention} in the annotations, further information would need be processed. This could be archived by user input, either from demanding formalized annotations or by text mining commit messages. 
Both are a imaginable features for a version of this diff-database project, adjusted for the use as personal VCS, opposed to initial statement.
Mentioned information could also be conceived by coupling this project with an existing VCS like Git or an data sharing platform like SEEK\footnote{\url{http://seek4science.org/}}.

\todo{find a nice ending?}
\todo{incorporate former outlook section}
\begin{itemize}
	\item outlook
	\subitem Improving search index with metrics on how much impact a change had to the search criteria
	\subsubitem "Incremental Query Evaluation. When a user has a standing query against a time-varying data source, a change-detection tool can provide the query engine the delta data on which the query will be re-evaluated. Thus, the user doesn’t receive old results and the query engine avoids repeated work. Since the delta data is usually much smaller than the original data, query evaluation will also be much faster." \citep{Wang2003}
	\subitem detecting similar changes on different models
\end{itemize}

\begin{comment}
\begin{itemize}
	\item Benchmarks/stats on database
		\subitem additional overhead (nodes/relations increase)
	\item How to improve database/reduce overhead
	\item 2 branches of \comodi unused
		\subitem because of automatic generation
		\subitem reason and intention not able to be automatically determined
		\subitem possible extension?
\end{itemize}
\end{comment}